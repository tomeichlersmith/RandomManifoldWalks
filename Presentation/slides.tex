
%% HUH Colloquium Presentation
%% Tom Eichlersmith

\documentclass{beamer}

\mode<presentation> {
	\usetheme{Goettingen}
	\setbeamertemplate{navigation symbols}{}
	\setbeamertemplate{footline}[page number]
}

\usepackage{graphicx} % Allows including images
\usepackage{booktabs} % Allows the use of \toprule, \midrule and \bottomrule in tables
\usepackage{subfigure} % For images next to each other
\usepackage{tikz} % Diagrams from dia

%-----------------------------------------------
%%	TITLE PAGE
%-----------------------------------------------

\title[Random Walks]{Random Walks on Simple Two-Dimensional Manifolds} % The short title appears at the bottom of every slide, the full title is only on the title page

\author{Tom Eichlersmith}
\institute[Hamline U]
{
Hamline University \\
\medskip
\texttt{teichlersmith01@hamline.edu}
}
\date{April 19, 2018}

\begin{document}

\begin{frame}
	\titlepage % Print the title page as the first slide
\end{frame}

%------------------------------------------------
%%	PRESENTATION SLIDES
%------------------------------------------------

\section{Introduction} 

\begin{frame}

	\frametitle{Introduction}
	
	\begin{itemize}
		\item Random
		\item Walk
		\item Simple
		\item Two-Dimensional
		\item Manifolds
	\end{itemize}

\end{frame}

%------------------------------------------------
\section{Background}

\begin{frame}
	
	\frametitle{Vocabulary}
	
	\begin{itemize}
		\item Smooth Surfaces
		\item Geodesic Equations
		\item Christoffel Symbols
	\end{itemize}
	
\end{frame}

\begin{frame}

	\frametitle{``Smooth" Surfaces}
	
	\begin{figure}
		\includegraphics[width=\textwidth]{images/Torus.png}
		\caption{By Leonid\_2 - Own work, CC BY-SA 3.0, \texttt{https://commons.wikimedia.org/w/index.php?curid=8643414}}
	\end{figure}
	
\end{frame}

%---------------------------------------------------------------

\begin{frame}
	
	\frametitle{Geodesic Equations}
	
	\begin{enumerate}
		\item Extend definition of line to other surfaces
		\item Assume a path is a geodesic contained in a coordinate patch
		\item Derive geodesic equations for coordinate functions of path
	\end{enumerate}

\end{frame}

%-------------------------------------------------

\begin{frame}
	
	\frametitle{Christoffel Symbols}
	
	\begin{itemize}
		\item Represent surface in geodesic equations
		\item Characterize properties of surface
	\end{itemize}
	
\end{frame}

%------------------------------------------------
\section{Method}

\begin{frame}
	
	\frametitle{C\texttt{++} Implementation}
	
	\begin{itemize}
		\item Runge-Kutta 4th Order Method
		\item Stack Linked List
		\item Function Pointers
	\end{itemize}
	
\end{frame}

%------------------------------------------------
\section{Results}

%Circular Ring

\begin{frame}
	
	\frametitle{Plane}
	
	\begin{figure}
		% Graphic for TeX using PGF
% Title: /home/tom/CodeProjects/MathDHP_201718/RandomManifoldWalks/Presentation/images/PlaneRing.dia
% Creator: Dia v0.97.3
% CreationDate: Mon Apr 23 12:42:23 2018
% For: tom
% \usepackage{tikz}
% The following commands are not supported in PSTricks at present
% We define them conditionally, so when they are implemented,
% this pgf file will use them.
\ifx\du\undefined
  \newlength{\du}
\fi
\setlength{\du}{15\unitlength}
\begin{tikzpicture}
\pgftransformxscale{1.000000}
\pgftransformyscale{-1.000000}
\definecolor{dialinecolor}{rgb}{0.000000, 0.000000, 0.000000}
\pgfsetstrokecolor{dialinecolor}
\definecolor{dialinecolor}{rgb}{1.000000, 1.000000, 1.000000}
\pgfsetfillcolor{dialinecolor}
\pgfsetlinewidth{0.100000\du}
\pgfsetdash{}{0pt}
\pgfsetdash{}{0pt}
\pgfsetbuttcap
\pgfsetmiterjoin
\pgfsetlinewidth{0.100000\du}
\pgfsetbuttcap
\pgfsetmiterjoin
\pgfsetdash{}{0pt}
\definecolor{dialinecolor}{rgb}{0.721569, 0.701961, 0.701961}
\pgfsetfillcolor{dialinecolor}
\pgfpathellipse{\pgfpoint{12.500000\du}{17.500000\du}}{\pgfpoint{7.500000\du}{0\du}}{\pgfpoint{0\du}{7.500000\du}}
\pgfusepath{fill}
\definecolor{dialinecolor}{rgb}{0.000000, 0.000000, 0.000000}
\pgfsetstrokecolor{dialinecolor}
\pgfpathellipse{\pgfpoint{12.500000\du}{17.500000\du}}{\pgfpoint{7.500000\du}{0\du}}{\pgfpoint{0\du}{7.500000\du}}
\pgfusepath{stroke}
\pgfsetbuttcap
\pgfsetmiterjoin
\pgfsetdash{}{0pt}
\definecolor{dialinecolor}{rgb}{0.000000, 0.000000, 0.000000}
\pgfsetstrokecolor{dialinecolor}
\pgfpathellipse{\pgfpoint{12.500000\du}{17.500000\du}}{\pgfpoint{7.500000\du}{0\du}}{\pgfpoint{0\du}{7.500000\du}}
\pgfusepath{stroke}
\pgfsetlinewidth{0.100000\du}
\pgfsetdash{}{0pt}
\pgfsetdash{}{0pt}
\pgfsetbuttcap
\pgfsetmiterjoin
\pgfsetlinewidth{0.100000\du}
\pgfsetbuttcap
\pgfsetmiterjoin
\pgfsetdash{}{0pt}
\definecolor{dialinecolor}{rgb}{1.000000, 1.000000, 1.000000}
\pgfsetfillcolor{dialinecolor}
\pgfpathellipse{\pgfpoint{12.500000\du}{17.500000\du}}{\pgfpoint{2.500000\du}{0\du}}{\pgfpoint{0\du}{2.500000\du}}
\pgfusepath{fill}
\definecolor{dialinecolor}{rgb}{0.000000, 0.000000, 0.000000}
\pgfsetstrokecolor{dialinecolor}
\pgfpathellipse{\pgfpoint{12.500000\du}{17.500000\du}}{\pgfpoint{2.500000\du}{0\du}}{\pgfpoint{0\du}{2.500000\du}}
\pgfusepath{stroke}
\pgfsetbuttcap
\pgfsetmiterjoin
\pgfsetdash{}{0pt}
\definecolor{dialinecolor}{rgb}{0.000000, 0.000000, 0.000000}
\pgfsetstrokecolor{dialinecolor}
\pgfpathellipse{\pgfpoint{12.500000\du}{17.500000\du}}{\pgfpoint{2.500000\du}{0\du}}{\pgfpoint{0\du}{2.500000\du}}
\pgfusepath{stroke}
\pgfsetlinewidth{0.000000\du}
\pgfsetdash{}{0pt}
\pgfsetdash{}{0pt}
\pgfsetbuttcap
\pgfsetmiterjoin
\pgfsetlinewidth{0.000000\du}
\pgfsetbuttcap
\pgfsetmiterjoin
\pgfsetdash{}{0pt}
\definecolor{dialinecolor}{rgb}{0.000000, 0.000000, 0.000000}
\pgfsetfillcolor{dialinecolor}
\pgfpathellipse{\pgfpoint{12.500000\du}{17.500000\du}}{\pgfpoint{0.100000\du}{0\du}}{\pgfpoint{0\du}{0.100000\du}}
\pgfusepath{fill}
\definecolor{dialinecolor}{rgb}{0.000000, 0.000000, 0.000000}
\pgfsetstrokecolor{dialinecolor}
\pgfpathellipse{\pgfpoint{12.500000\du}{17.500000\du}}{\pgfpoint{0.100000\du}{0\du}}{\pgfpoint{0\du}{0.100000\du}}
\pgfusepath{stroke}
\pgfsetbuttcap
\pgfsetmiterjoin
\pgfsetdash{}{0pt}
\definecolor{dialinecolor}{rgb}{0.000000, 0.000000, 0.000000}
\pgfsetstrokecolor{dialinecolor}
\pgfpathellipse{\pgfpoint{12.500000\du}{17.500000\du}}{\pgfpoint{0.100000\du}{0\du}}{\pgfpoint{0\du}{0.100000\du}}
\pgfusepath{stroke}
\pgfsetlinewidth{0.100000\du}
\pgfsetdash{}{0pt}
\pgfsetdash{}{0pt}
\pgfsetbuttcap
{
\definecolor{dialinecolor}{rgb}{0.000000, 0.000000, 0.000000}
\pgfsetfillcolor{dialinecolor}
% was here!!!
\definecolor{dialinecolor}{rgb}{0.000000, 0.000000, 0.000000}
\pgfsetstrokecolor{dialinecolor}
\draw (12.500000\du,17.500000\du)--(12.500000\du,17.500000\du);
}
\pgfsetlinewidth{0.100000\du}
\pgfsetdash{}{0pt}
\pgfsetdash{}{0pt}
\pgfsetbuttcap
{
\definecolor{dialinecolor}{rgb}{0.000000, 0.000000, 0.000000}
\pgfsetfillcolor{dialinecolor}
% was here!!!
\definecolor{dialinecolor}{rgb}{0.000000, 0.000000, 0.000000}
\pgfsetstrokecolor{dialinecolor}
\draw (12.500000\du,17.500000\du)--(12.500000\du,17.500000\du);
}
\pgfsetlinewidth{0.100000\du}
\pgfsetdash{}{0pt}
\pgfsetdash{}{0pt}
\pgfsetbuttcap
{
\definecolor{dialinecolor}{rgb}{0.000000, 0.000000, 0.000000}
\pgfsetfillcolor{dialinecolor}
% was here!!!
\definecolor{dialinecolor}{rgb}{0.000000, 0.000000, 0.000000}
\pgfsetstrokecolor{dialinecolor}
\draw (12.600000\du,17.500000\du)--(12.500000\du,17.500000\du);
}
\pgfsetlinewidth{0.100000\du}
\pgfsetdash{}{0pt}
\pgfsetdash{}{0pt}
\pgfsetbuttcap
{
\definecolor{dialinecolor}{rgb}{0.000000, 0.000000, 0.000000}
\pgfsetfillcolor{dialinecolor}
% was here!!!
\definecolor{dialinecolor}{rgb}{0.000000, 0.000000, 0.000000}
\pgfsetstrokecolor{dialinecolor}
\draw (12.500000\du,17.500000\du)--(12.500000\du,17.500000\du);
}
\end{tikzpicture}

	\end{figure}
	
\end{frame}

\begin{frame}

	\frametitle{Plane}
	
	\begin{figure}
		\includegraphics[width=0.85\textwidth]{images/PlaneIn1Out3.pdf}
	\end{figure}

\end{frame}

%---------------------------------------------------------------

%Circular Walls

\begin{frame}
	
	\frametitle{Plane}
	
	\begin{figure}
		% Graphic for TeX using PGF
% Title: /home/tom/CodeProjects/MathDHP_201718/RandomManifoldWalks/Presentation/images/PlaneWalls.dia
% Creator: Dia v0.97.3
% CreationDate: Mon Apr 23 12:42:27 2018
% For: tom
% \usepackage{tikz}
% The following commands are not supported in PSTricks at present
% We define them conditionally, so when they are implemented,
% this pgf file will use them.
\ifx\du\undefined
  \newlength{\du}
\fi
\setlength{\du}{15\unitlength}
\begin{tikzpicture}
\pgftransformxscale{1.000000}
\pgftransformyscale{-1.000000}
\definecolor{dialinecolor}{rgb}{0.000000, 0.000000, 0.000000}
\pgfsetstrokecolor{dialinecolor}
\definecolor{dialinecolor}{rgb}{1.000000, 1.000000, 1.000000}
\pgfsetfillcolor{dialinecolor}
\pgfsetlinewidth{0.100000\du}
\pgfsetdash{}{0pt}
\pgfsetdash{}{0pt}
\pgfsetbuttcap
\pgfsetmiterjoin
\pgfsetlinewidth{0.100000\du}
\pgfsetbuttcap
\pgfsetmiterjoin
\pgfsetdash{}{0pt}
\definecolor{dialinecolor}{rgb}{0.721569, 0.701961, 0.701961}
\pgfsetfillcolor{dialinecolor}
\pgfpathellipse{\pgfpoint{10.000000\du}{15.000000\du}}{\pgfpoint{7.500000\du}{0\du}}{\pgfpoint{0\du}{7.500000\du}}
\pgfusepath{fill}
\definecolor{dialinecolor}{rgb}{0.000000, 0.000000, 1.000000}
\pgfsetstrokecolor{dialinecolor}
\pgfpathellipse{\pgfpoint{10.000000\du}{15.000000\du}}{\pgfpoint{7.500000\du}{0\du}}{\pgfpoint{0\du}{7.500000\du}}
\pgfusepath{stroke}
\pgfsetbuttcap
\pgfsetmiterjoin
\pgfsetdash{}{0pt}
\definecolor{dialinecolor}{rgb}{0.000000, 0.000000, 1.000000}
\pgfsetstrokecolor{dialinecolor}
\pgfpathellipse{\pgfpoint{10.000000\du}{15.000000\du}}{\pgfpoint{7.500000\du}{0\du}}{\pgfpoint{0\du}{7.500000\du}}
\pgfusepath{stroke}
\pgfsetlinewidth{0.100000\du}
\pgfsetdash{}{0pt}
\pgfsetdash{}{0pt}
\pgfsetbuttcap
\pgfsetmiterjoin
\pgfsetlinewidth{0.100000\du}
\pgfsetbuttcap
\pgfsetmiterjoin
\pgfsetdash{}{0pt}
\definecolor{dialinecolor}{rgb}{1.000000, 1.000000, 1.000000}
\pgfsetfillcolor{dialinecolor}
\fill (8.032258\du,13.000000\du)--(8.032258\du,17.058333\du)--(11.959677\du,17.058333\du)--(11.959677\du,13.000000\du)--cycle;
\definecolor{dialinecolor}{rgb}{0.000000, 0.000000, 0.000000}
\pgfsetstrokecolor{dialinecolor}
\draw (8.032258\du,13.000000\du)--(8.032258\du,17.058333\du)--(11.959677\du,17.058333\du)--(11.959677\du,13.000000\du)--cycle;
\pgfsetbuttcap
\pgfsetmiterjoin
\pgfsetdash{}{0pt}
\definecolor{dialinecolor}{rgb}{0.000000, 0.000000, 0.000000}
\pgfsetstrokecolor{dialinecolor}
\draw (8.032258\du,13.000000\du)--(8.032258\du,17.058333\du)--(11.959677\du,17.058333\du)--(11.959677\du,13.000000\du)--cycle;
\pgfsetlinewidth{0.000000\du}
\pgfsetdash{}{0pt}
\pgfsetdash{}{0pt}
\pgfsetbuttcap
\pgfsetmiterjoin
\pgfsetlinewidth{0.000000\du}
\pgfsetbuttcap
\pgfsetmiterjoin
\pgfsetdash{}{0pt}
\definecolor{dialinecolor}{rgb}{0.000000, 0.000000, 0.000000}
\pgfsetfillcolor{dialinecolor}
\pgfpathellipse{\pgfpoint{10.000000\du}{15.000000\du}}{\pgfpoint{0.100000\du}{0\du}}{\pgfpoint{0\du}{0.100000\du}}
\pgfusepath{fill}
\definecolor{dialinecolor}{rgb}{0.000000, 0.000000, 0.000000}
\pgfsetstrokecolor{dialinecolor}
\pgfpathellipse{\pgfpoint{10.000000\du}{15.000000\du}}{\pgfpoint{0.100000\du}{0\du}}{\pgfpoint{0\du}{0.100000\du}}
\pgfusepath{stroke}
\pgfsetbuttcap
\pgfsetmiterjoin
\pgfsetdash{}{0pt}
\definecolor{dialinecolor}{rgb}{0.000000, 0.000000, 0.000000}
\pgfsetstrokecolor{dialinecolor}
\pgfpathellipse{\pgfpoint{10.000000\du}{15.000000\du}}{\pgfpoint{0.100000\du}{0\du}}{\pgfpoint{0\du}{0.100000\du}}
\pgfusepath{stroke}
\end{tikzpicture}

	\end{figure}
	
\end{frame}

\begin{frame}
	
	\frametitle{Plane}
	
	\begin{figure}
		\includegraphics[width=0.85\textwidth]{images/PlaneCircleL05.pdf}
	\end{figure}
	
\end{frame}

%---------------------------------------------------------------

%Sphere Polar Escape Circle

\begin{frame}
	
	\frametitle{Sphere}
	
	\begin{figure}
		% Graphic for TeX using PGF
% Title: /home/tom/CodeProjects/MathDHP_201718/RandomManifoldWalks/Presentation/images/SpherePolar.dia
% Creator: Dia v0.97.3
% CreationDate: Mon Apr 23 12:42:31 2018
% For: tom
% \usepackage{tikz}
% The following commands are not supported in PSTricks at present
% We define them conditionally, so when they are implemented,
% this pgf file will use them.
\ifx\du\undefined
  \newlength{\du}
\fi
\setlength{\du}{12\unitlength}
\begin{tikzpicture}
\pgftransformxscale{1.000000}
\pgftransformyscale{-1.000000}
\definecolor{dialinecolor}{rgb}{0.000000, 0.000000, 0.000000}
\pgfsetstrokecolor{dialinecolor}
\definecolor{dialinecolor}{rgb}{1.000000, 1.000000, 1.000000}
\pgfsetfillcolor{dialinecolor}
\pgfsetlinewidth{0.100000\du}
\pgfsetdash{}{0pt}
\pgfsetdash{}{0pt}
\pgfsetbuttcap
\pgfsetmiterjoin
\pgfsetlinewidth{0.100000\du}
\pgfsetbuttcap
\pgfsetmiterjoin
\pgfsetdash{}{0pt}
\definecolor{dialinecolor}{rgb}{1.000000, 1.000000, 1.000000}
\pgfsetfillcolor{dialinecolor}
\pgfpathellipse{\pgfpoint{9.950000\du}{12.450000\du}}{\pgfpoint{7.450000\du}{0\du}}{\pgfpoint{0\du}{7.450000\du}}
\pgfusepath{fill}
\definecolor{dialinecolor}{rgb}{0.000000, 0.000000, 0.000000}
\pgfsetstrokecolor{dialinecolor}
\pgfpathellipse{\pgfpoint{9.950000\du}{12.450000\du}}{\pgfpoint{7.450000\du}{0\du}}{\pgfpoint{0\du}{7.450000\du}}
\pgfusepath{stroke}
\pgfsetbuttcap
\pgfsetmiterjoin
\pgfsetdash{}{0pt}
\definecolor{dialinecolor}{rgb}{0.000000, 0.000000, 0.000000}
\pgfsetstrokecolor{dialinecolor}
\pgfpathellipse{\pgfpoint{9.950000\du}{12.450000\du}}{\pgfpoint{7.450000\du}{0\du}}{\pgfpoint{0\du}{7.450000\du}}
\pgfusepath{stroke}
\pgfsetlinewidth{0.100000\du}
\pgfsetdash{}{0pt}
\pgfsetdash{}{0pt}
\pgfsetbuttcap
{
\definecolor{dialinecolor}{rgb}{0.000000, 0.000000, 0.000000}
\pgfsetfillcolor{dialinecolor}
% was here!!!
\definecolor{dialinecolor}{rgb}{0.000000, 0.000000, 0.000000}
\pgfsetstrokecolor{dialinecolor}
\draw (9.950000\du,12.450000\du)--(9.950000\du,12.450000\du);
}
\pgfsetlinewidth{0.100000\du}
\pgfsetdash{{1.000000\du}{1.000000\du}}{0\du}
\pgfsetdash{{1.000000\du}{1.000000\du}}{0\du}
\pgfsetbuttcap
{
\definecolor{dialinecolor}{rgb}{0.000000, 0.000000, 0.000000}
\pgfsetfillcolor{dialinecolor}
% was here!!!
\definecolor{dialinecolor}{rgb}{0.000000, 0.000000, 0.000000}
\pgfsetstrokecolor{dialinecolor}
\draw (9.950000\du,12.450000\du)--(9.950000\du,12.450000\du);
}
\pgfsetlinewidth{0.100000\du}
\pgfsetdash{{1.000000\du}{1.000000\du}}{0\du}
\pgfsetdash{{1.000000\du}{1.000000\du}}{0\du}
\pgfsetbuttcap
{
\definecolor{dialinecolor}{rgb}{0.000000, 0.000000, 0.000000}
\pgfsetfillcolor{dialinecolor}
% was here!!!
\definecolor{dialinecolor}{rgb}{0.000000, 0.000000, 0.000000}
\pgfsetstrokecolor{dialinecolor}
\draw (9.950000\du,12.450000\du)--(9.950000\du,12.450000\du);
}
\pgfsetlinewidth{0.100000\du}
\pgfsetdash{{1.000000\du}{1.000000\du}}{0\du}
\pgfsetdash{{1.000000\du}{1.000000\du}}{0\du}
\pgfsetbuttcap
{
\definecolor{dialinecolor}{rgb}{0.000000, 0.000000, 0.000000}
\pgfsetfillcolor{dialinecolor}
% was here!!!
\definecolor{dialinecolor}{rgb}{0.000000, 0.000000, 0.000000}
\pgfsetstrokecolor{dialinecolor}
\draw (9.950000\du,12.450000\du)--(9.950000\du,12.450000\du);
}
\pgfsetlinewidth{0.100000\du}
\pgfsetdash{{1.000000\du}{1.000000\du}}{0\du}
\pgfsetdash{{1.000000\du}{1.000000\du}}{0\du}
\pgfsetbuttcap
{
\definecolor{dialinecolor}{rgb}{0.000000, 0.000000, 0.000000}
\pgfsetfillcolor{dialinecolor}
% was here!!!
\definecolor{dialinecolor}{rgb}{0.000000, 0.000000, 0.000000}
\pgfsetstrokecolor{dialinecolor}
\draw (9.950000\du,12.450000\du)--(9.950000\du,12.450000\du);
}
\pgfsetlinewidth{0.100000\du}
\pgfsetdash{{\pgflinewidth}{0.200000\du}}{0cm}
\pgfsetdash{{\pgflinewidth}{0.200000\du}}{0cm}
\pgfsetbuttcap
{
\definecolor{dialinecolor}{rgb}{0.000000, 0.000000, 0.000000}
\pgfsetfillcolor{dialinecolor}
% was here!!!
\definecolor{dialinecolor}{rgb}{0.000000, 0.000000, 0.000000}
\pgfsetstrokecolor{dialinecolor}
\draw (9.900000\du,2.800000\du)--(9.950000\du,21.300000\du);
}
\pgfsetlinewidth{0.100000\du}
\pgfsetdash{}{0pt}
\pgfsetdash{}{0pt}
\pgfsetbuttcap
{
\definecolor{dialinecolor}{rgb}{0.000000, 0.000000, 0.000000}
\pgfsetfillcolor{dialinecolor}
% was here!!!
\definecolor{dialinecolor}{rgb}{0.000000, 0.000000, 0.000000}
\pgfsetstrokecolor{dialinecolor}
\pgfpathmoveto{\pgfpoint{3.998715\du}{7.999559\du}}
\pgfpatharc{109}{72}{18.500000\du and 18.500000\du}
\pgfusepath{stroke}
}
\pgfsetlinewidth{0.050000\du}
\pgfsetdash{}{0pt}
\pgfsetdash{}{0pt}
\pgfsetbuttcap
{
\definecolor{dialinecolor}{rgb}{0.000000, 0.000000, 0.000000}
\pgfsetfillcolor{dialinecolor}
% was here!!!
\pgfsetarrowsstart{stealth}
\definecolor{dialinecolor}{rgb}{0.000000, 0.000000, 0.000000}
\pgfsetstrokecolor{dialinecolor}
\draw (16.025000\du,8.056250\du)--(9.925000\du,12.050000\du);
}
\pgfsetlinewidth{0.050000\du}
\pgfsetdash{}{0pt}
\pgfsetdash{}{0pt}
\pgfsetbuttcap
{
\definecolor{dialinecolor}{rgb}{0.000000, 0.000000, 0.000000}
\pgfsetfillcolor{dialinecolor}
% was here!!!
\definecolor{dialinecolor}{rgb}{0.000000, 0.000000, 0.000000}
\pgfsetstrokecolor{dialinecolor}
\pgfpathmoveto{\pgfpoint{16.025048\du}{8.006314\du}}
\pgfpatharc{324}{217}{7.481389\du and 7.481389\du}
\pgfusepath{stroke}
}
\end{tikzpicture}

	\end{figure}
	
\end{frame}

\begin{frame}
	
	\frametitle{Sphere}
	
	\begin{figure}
		\includegraphics[width=0.85\textwidth]{images/ExampleSphereL04.pdf}
	\end{figure}
	
\end{frame}

%---------------------------------------------------------------

\begin{frame}
	
	\frametitle{Sphere}
	
	\begin{figure}
		\includegraphics[width=0.85\textwidth]{images/SummaryPlot_L005_04.pdf}
	\end{figure}
	
\end{frame}

%------------------------------------------------
\section{Questions}

\begin{frame}
	
	\Huge{\centerline{Questions?}}
	
\end{frame}

\end{document} 