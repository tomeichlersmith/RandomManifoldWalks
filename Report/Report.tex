\documentclass{article}

\usepackage[margin=1in]{geometry}
\usepackage{amsfonts, amsmath}

\newcommand{\R}{\mathbb{R}}

\title{Random Walks on Simple Two-Dimensional Manifolds}
\author{Tom Eichlersmith \\ Art Guetter (adv)}

\begin{document}

\section{Introduction}
	For us to be capable of exploring the central problem explained later in this paper, we must develop machinery in order to replicate the intuitive notion of a walk on different surfaces.
	Most of this machinery has been developed in the field of differential geometry and we have learned a lot of this material from \cite{BanchoffLovett_DiffGeo_2010} (Read it for a more detailed explanation of this material).
	We will begin with a general overview of differential manifolds that are locally Euclidean.
	
	\subsection{Regular Surfaces}
		The introduction to parametrized surfaces given in \cite{BanchoffLovett_DiffGeo_2010} is most readily adapted to this project because it can be easily implemented in computer programming.
		We construct a surface as a subset of $\R^3$ by considering continuous functions $\mu:U \to V$ where $U \subseteq \R^2$ and $V$ is a subset of the surface --- we call the set of $\mu$ that covers the entire surface in question a \textit{chart}.
		Glossing over the large amount of detail and development that can be extracted from this simple idea of a surface, we apply more restrictions on the surfaces (and their charts) in order to obtain an intuitive "smoothness" on the surface.
		In a sense, these restrictions allow us to consider surfaces that are locally Euclidean and therefore they more closely resemble what we perceive a surface to be (hence the name for this class of surfaces).
		For a surface to be regular, there must exist a chart that has a member $\mu$ covering an open neighborhood around each point of the surface and satisfies the following conditions:
		\begin{enumerate}
			\item Differentiable --- the coordinate functions of $\mu$ in $\R^3$ have continuous partial derivatives for all orders
			\item Homeomorphic --- $\mu$ and its inverse are continuous
			\item Satisfies the Regularity Condition --- The differential of $\mu$ is a one-to-one linear transformation
		\end{enumerate}
		Often, a member of the chart for a regular surface is called a \textit{coordinate patch} because it imposes a homeomorphic relationship between coordinates in $\R^2$ --- represented by $U$ --- onto a "patch" enclosing an open neighborhood on the surface --- represented by $V$.
		When properly constructed, a chart of a regular surface completely characterizes it, and we can calculate all of the properties of the surface we require from the chart.
		

\section{Derivation of Geodesic Equations}
	
	\subsection{Christoffel Symbols}
		In order for the implementation of the geodesic equations in C\texttt{++} to be simpler, we change the domain of the standard charts for the sphere and the torus to be the unit square $I^2 = [0,1) \times [0,1)$.
		This will yield slightly different Christoffel Symbols for both of these manifolds (and their charts).
		First, we will calculate the Christoffel Symbols for the sphere $S^2$ mapped with the chart $\mu:I^2 \to S^2$ defined by
		$$ \mu(u,v) = (\cos(2\pi u)\sin(\pi v), \sin(2\pi u)\cos(\pi v), \cos(\pi v)) $$
		which has the corresponding Christoffel Symbols
		\begin{equation}
			\begin{array}{lr}
			\left(\Gamma^{u}_{ij}\right) = \left( \begin{array}{cc}
					0 & \pi\cot(\pi v) \\
					\pi\cot(\pi v) & 0
				\end{array} \right) &
			\left(\Gamma^{v}_{ij}\right) = \left( \begin{array}{cc}
					-2\pi\sin(2\pi v) & 0 \\
					0 & 0
				\end{array} \right)
			\end{array}
		\end{equation}
	
\section{Random Walks}
	
	\subsection{Optimizations}

\section{Numeric Solution}

\section{Results}

\bibliographystyle{plain}
\bibliography{../../References/References_MathDHP_201718}

\end{document}