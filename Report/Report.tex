\documentclass{article}

\usepackage[margin=1in]{geometry}

\title{Random Walks on Simple Two-Dimensional Manifolds}
\author{Tom Eichlersmith \\ Art Guetter (adv)}

\begin{document}

\section{Introduction}
	For us to be capable of exploring the central problem explained later in this paper, we must develop machinery in order to replicate the intuitive notion of a walk on different surfaces.
	Most of this machinery has been developed in the field of differential geometry and we have learned a lot of this material from \cite{BanchoffLovett_DiffGeo_2010} (Read it for a more detailed explanation of this material).
	We will begin with a general overview of differential manifolds that are locally Euclidean.
	
	\subsection{Euclidean Manifolds}
		

\section{Derivation of Geodesic Equations}
	
	\subsection{Christoffel Symbols}
		In order for the implementation of the geodesic equations in C\texttt{++} to be simpler, we change the domain of the standard charts for the sphere and the torus to be the unit square $I^2 = [0,1) \times [0,1)$.
		This will yield slightly different Christoffel Symbols for both of these manifolds (and their charts).
		First, we will calculate the Christoffel Symbols for the sphere $S^2$ mapped with the chart $\mu:I^2 \to S^2$ defined by
		$$ \mu(u,v) = (\cos(2\pi u)\sin(\pi v), \sin(2\pi u)\cos(\pi v), \cos(\pi v)) $$
		which has the corresponding Christoffel Symbols
		\begin{equation}
			\begin{array}{lr}
			\left(\Gamma^{u}_{ij}\right) = \left( \begin{array}{cc}
					0 & \pi\cot(\pi v) \\
					\pi\cot(\pi v) & 0
				\end{array} \right) &
			\left(\Gamma^{v}_{ij}\right) = \left( \begin{array}{cc}
					-2\pi\sin(2\pi v) & 0 \\
					0 & 0
				\end{array} \right)
			\end{array}
		\end{equation}
	
\section{Random Walks}
	
	\subsection{Optimizations}

\section{Numeric Solution}

\section{Results}

\bibliographystyle{plain}
\bibliography{../../References/References_MathDHP_201718}

\end{document}